% ----------------------------------------------------
% Magic Comments for LaTeX-editors (TeXstudio, VSCode with LaTeX Workshop extension...)
% !TeX encoding = utf8
% !TeX spellcheck = de_DE
% !TeX program = pdflatex
% !BIB program = biber
% ----------------------------------------------------

% ----------------------------------------------------
% Load the HfTL-Thesis class
\documentclass[
% Select the main Language of your thesis here. Titlepage, captions and statement of authorship will change accordingly:
%
ngerman % - Deutsch (default)
% USenglish % - American english
% UKenglish % - British english
%
numeric % (default) use numeric citation style sorted by the occurence in text: [5]
% alphabetic % - use alphabetic citation style: [Lau95]
% authoryear % - use authoryear citation style: Laubach 1995
]{wbh-assignment}
% ----------------------------------------------------

% ----------------------------------------------------
% Load your own packages here
\usepackage{amsmath}
\usepackage{amsfonts}
\usepackage{amssymb}
\usepackage{graphicx}
\usepackage{scrlayer}
\usepackage{tabularx}
\usepackage{geometry}
\usepackage{setspace}
\usepackage[right]{eurosym}
%\usepackage[printonlyused]{acronym}
\usepackage{subfig}
\usepackage{floatflt}
% \usepackage[usenames,dvipsnames]{color}
\usepackage{colortbl}
\usepackage{paralist}
\usepackage{array}
%\usepackage{titlesec}
% \usepackage{parskip}
\usepackage[right]{eurosym}
%\usepackage{wrapfig}
% \usepackage[subfigure,titles]{tocloft}
\usepackage{helvet}
\usepackage{XCharter}
\usepackage{listings}
\usepackage{xcolor}
\usepackage{tcolorbox}
% \usepackage{enumitem}
% \usepackage{tikz}
% \usetikzlibrary{automata, positioning, arrows}

% Definition der gewünschten Rahmenfarbe mit Hex-Code
\definecolor{rahmenfarbe}{HTML}{d7005f}

% Definition des tcolorbox-Umfelds für Aufgabenstellungen
\newtcolorbox{aufgabenstellung}[1][]{%
  colback=white,  % Hintergrundfarbe
  colframe=rahmenfarbe,   % Rahmenfarbe
  title=Aufgabenstellung, % Titel der Box
  sharp corners,
  #1
}

\definecolor{ao(english)}{rgb}{0.0, 0.5, 0.0}

\lstset{
%	language=Prolog,
	basicstyle=\footnotesize,
	keywordstyle=\color{blue},
	commentstyle=\color{gray},
	stringstyle=\color{ao(english)},
	numbers=left,
	numberstyle=\tiny\color{gray},
	stepnumber=1,
	numbersep=5pt,
	backgroundcolor=\color{lightgray!20},
	frame=single,
	tabsize=2,
	captionpos=b,
	breaklines=true,
	breakatwhitespace=false,
	showspaces=false,
	showstringspaces=false,
	showtabs=false,
	morekeywords={:-},
}

% ----------------------------------------------------

% ----------------------------------------------------
% The document body. Start your work here.
\geometry{a4paper, top=25mm, left=30mm, right=40mm, bottom=25mm, headsep=10mm, footskip=12mm}

\renewcommand{\familydefault}{\sfdefault}

% ----------------------------------------------------
% Include own references
% \addbibresource{References/online.bib}
% \addbibresource{References/3gpp.bib}
% \addbibresource{References/etsi.bib}
% \addbibresource{References/itut.bib}
% \addbibresource{References/rfc.bib}
% \addbibresource{References/zotero.bib}
% \addbibresource{References/new.bib}
% \addbibresource{quellen.bib}
\addbibresource{references.bib}
\addbibresource{References/Embedded Software Engineering.bib}

% ----------------------------------------------------

% ----------------------------------------------------
% Load acronym definitions
%Eigene Definitionen
\newacronym{ftth}{FTTH}{Fiber To The Home}
\newacronym{fttc}{FTTC}{Fibre To The Curb}
\newacronym{fttb}{FTTB}{Fibre To The Building}
\newacronym{fttd}{FTTD}{Fibre To The Desk}
\newacronym{tkg}{TKG}{Telekommunikationsgesetz}
\newacronym{mbdc}{MBDC}{Minimum bidirectional net data rate capability}
\newacronym{pti}{PTI}{Produktion Technische Infrastruktur}
\newacronym{wfm t}{WFM T}{Workforce Management Technik}
\newacronym{kvz}{KVz}{Kabelverzweiger}
\newacronym{apl}{APL}{Abschlusspunkt Linientechnik}
\newacronym{mfg}{MFG}{Multifunktionsgehäuse}
\newacronym{nvt}{NVt}{Netzverteiler}
\newacronym{sve}{SVE}{Stromversorgungseinheit}
\newacronym{epk}{EPK}{Ereignisgesteuerte Prozesskette}
\newacronym{eepk}{eEPK}{erweiterte Ereignisgesteuerte Prozesskette}
\newacronym{epc}{EPC}{Event-driven Process Chain}
\newacronym{bpmn}{BPMN}{Business Process Model and Notation}
\newacronym{hk}{Hk}{Hauptkabel}
\newacronym{vzk}{Vzk}{Verzweigungskabel}
\newacronym{qk}{Qk}{Querkabel}
\newacronym{ovst}{OVSt}{Ortsvermittlungsstellen}
\newacronym{hvt}{HVt}{Hauptverteiler}
\newacronym{exif}{Exif}{Exchangeable Image File Format}
\newacronym{5g}{5G}{fünfte Generation [des Mobilfunks]}
\newacronym{gps}{GPS}{Global Positioning System}
\newacronym{onkz}{ONKZ}{Ortsnetzkennzahl}
\newacronym{asb}{ASB}{Anschlussbereich}
\newacronym{psl}{PSL}{Produktions-, Service-, Logistikprozesse}
\newacronym{pwa}{PWA}{Progressive Web App}
\newacronym{dsl}{DSL}{Digital Subscriber Line}
\newacronym{adsl}{ADSL}{Asymmetric Digital Subscriber Line}
\newacronym{hdsl}{HDSL}{High-Bit-Rate Digital Subscriber Line}
\newacronym{sdsl}{SDSL}{Symmetric Digital Subscriber Line}
\newacronym{vdsl}{VDSL}{Very-High-Bit-Rate Digital Subscriber Line}
\newacronym{fttx}{FTTx}{Fibre to the X}
\newacronym{diginetz}{DigiNetz}{Gesetz zur Erleichterung des Ausbaus digitaler Hochgeschwindigkeitsnetze}
\newacronym{atb-bestra}{ATB-BeStra}{Allgemeinen technischen Bestimmungen für die Benutzung von Strassen durch Leitungen und Telekommunikationslinien}
\newacronym{isdn}{ISDN}{Integrated Services Digital Network}
\newacronym{isdn d}{ISDN}{Integriertes Sprach- und Datennetz}
\newacronym{cn}{CN}{Corporate Network}
\newacronym{aps}{APS}{Arbeitsplatzsystem}
\newacronym{enas}{EnAS}{Energie-Anschluss-Säule}
\newacronym{bmvi}{BMVI}{Bundesministerium für Verkehr und digitale Infrastruktur}
\newacronym{gpon}{GPON}{Gigabit Passive Optical Network}
\newacronym{ptp}{PtP}{PtP-Ethernet}
\newacronym{olt}{OLT}{Optical Line Termination}
\newacronym{onu}{ONU}{Optical Network Unit}
\newacronym{ztv-tknetz}{ZTV-TKNetz}{Zusätzlichen Technischen Vertragsbedingungen der
Deutschen Telekom AG für Bauleistungen am Telekommunikationsnetz}
\newacronym{fstrg}{FStrG}{Bundesfernstraßengesetz}
\newacronym{dtag}{DTAG}{Deutschen Telekom AG}
\newacronym{mbfd}{MBfD}{Mehr Breitband für Deutschland}
\newacronym{html}{HTML}{HyperText Markup Language}
\newacronym{css}{CSS}{Cascading Style Sheets}
\newacronym{https}{HTTPS}{Hypertext Transfer Protocol Secure}

%Aus der Vorlage
\newacronym{3gpp}{3GPP}{Third Generation Partnership Project}

\newacronym{aaa}{AAA}{Authentication, Authorisation and Accounting}
\newacronym{aac-eld}{AAC-ELD}{Advanced Audio Coding-Enhanced Low Delay}
\newacronym{acr}{ACR}{Absolute Category Rating}
\newacronym{af}{AF}{Application Function}
\newacronym{agcf}{AGCF}{Access Gateway Control Function}
\newacronym{agch}{AGCH}{Access Grant Channel}
\newacronym{ajax}{AJAX}{Asynchronous JavaScript and XML}
\newacronym{amr}{AMR}{Adaptive Multi Rate}
\newacronym{amr-wb}{AMR-WB}{Adaptive Multirate-Wideband}
\newacronym{ape}{APE}{AJAX Push Engine}
\newacronym{api}{API}{Application Programming Interface}
%\acroplural{api}[APIs]{Application Programming Interfaces}
\newacronym{apn}{APN}{Access Point Name}
\newacronym{appserv}{AS}{Application Server}
\newacronym{armgw}{A/R-MGW}{Access/Residential-Media Gateway}
\newacronym{arvgw}{A/R-VGW}{Access/Residential-Voice over IP-Gateway}
\newacronym{as}{AS}{Access Stratum}
\newacronym{auc}{AuC}{Authentication Centre}
\newacronym{avp}{AVP}{Attribute-Value-Pairs}
\newacronym{ar}{AR}{Argumented Reality}
\newacronym{ad}{AD}{Activity Diagram}

\newacronym{bcch}{BCCH}{Broadcast Control Channel}
\newacronym{bgcf}{BGCF}{Breakout Gateway Control Function}
\newacronym{bsc}{BSC}{Base Station Controller}
\newacronym{bts}{BTS}{Base Transceiver Station}

\newacronym{ca}{CA}{Certificate Authority}
\newacronym{ca_mac}{CA}{Collision Avoidance}
\newacronym{camel}{CAMEL}{Customised Application for Mobile network Enhanced Logic}
\newacronym{cc}{CC}{Call Control, Country Code}
\newacronym{celp}{CELP}{Code-Excited Linear-Prediction}
\newacronym{celt}{CELT}{Constrained Energy Lapped Transform}
\newacronym{cli}{CLI}{Command Line Interface}
\newacronym{cm}{CM}{Connection Management}
\newacronym{cng}{CNG}{Comfort Noise Generation}
\newacronym{cod}{CoD}{Content on Demand}
\newacronym{cp}{CP}{Control Plane}
\newacronym{cpe}{CPE}{Customer-Premises Equipment}
\newacronym{cs}{CS}{Circuit Switched}
\newacronym{cs-acelp}{CS-ACELP}{Conjugate-Structure Algebraic-Code Excited Linear-Prediction}
\newacronym{cscf}{CSCF}{Call/Session Control Function}
\newacronym{csfb}{CSFB}{Circuit Switched Fallback}
\newacronym{csma}{CSMA}{Carrier Sense Medium Access}
%\newacronym{css}{CSS}{Cascading Style Sheets}
\newacronym{cps}{CPS}{Cyber-physische Systeme}

\newacronym{db}{DB}{Database}
\newacronym{dom}{DOM}{Document Object Model}
\newacronym{dscp}{DSCP}{Differentiated Services Code Points}
\newacronym{dtls}{DTLS}{Datagram Transport Layer Security}
\newacronym{dtx}{DTX}{Discontinuous Transmission}
\newacronym{dv}{DV}{Delay Variation}
\newacronym{dsgvo}{DSGVO}{Datenschutz-Grundverordnung}

\newacronym{e-rab}{E-RAB}{E-UTRAN Radio Access Bearer}
\newacronym{e-utran}{E-UTRAN}{Evolved UMTS Radio Access Network}
\newacronym{e2e}{E2E}{End-to-End}
\newacronym{edge}{EDGE}{Enhanced Data rates for GSM Evolution}
\newacronym{edss1}{EDSS1}{European Digital Subscriber System No. 1}
\newacronym{eir}{EIR}{Equipment Identity Centre, Equipment Identity Register}
%\newacronym{epc}{EPC}{Evolved Packet Core}
\newacronym{epg}{EPG}{Electronic Program Guide}
\newacronym{eps}{EPS}{Evolved Packet System}
\newacronym{etsi}{ETSI}{European Telecommunication Standards Institute}
\newacronym{etsi-tispan}{ETSI-TISPAN}{ETSI-Telecommunications and Internet converged Services and Protocols for AdvancedNetworking}

\newacronym{facch}{FACCH}{Fast Associated Control CHannel}
\newacronym{fb}{FB}{Fullband}
\newacronym{fcch}{FCCH}{Frequency Correction CHannel}
\newacronym{fec}{FEC}{Forward Error Correction}
\newacronym{fmc}{FMC}{Fixed Mobile Convergence}
\newacronym{fokus}{FOKUS}{Fraunhofer-Institut für Offene Kommunikationssysteme}
\newacronym{fr}{FR}{Full Rate}

\newacronym{geran}{GERAN}{GSM EDGE Radio Access Network}
\newacronym{ggsn}{GGSN}{Gateway GPRS Support Node}
\newacronym{gmm}{GMM}{GPRS Mobility Management}
\newacronym{gmsc}{GMSC}{Gateway MSC}
\newacronym{gprs}{GPRS}{General Packet Radio Service}
\newacronym{gsm}{GSM}{Global System for Mobile communications}
\newacronym{gsma}{GSMA}{GSM Association}
\newacronym{gtp}{GTP}{GPRS Tunneling Protocol}
\newacronym{gui}{GUI}{Graphical User Interface}
\newacronym{gw}{GW}{Gateway}

\newacronym{hd}{HD}{High Definition}
\newacronym{hlr}{HLR}{Home Location Register}
\newacronym{hr}{HR}{Half Rate}
\newacronym{hscsd}{HSCSD}{High-Speed Circuit-Switched Data}
\newacronym{hss}{HSS}{Home Subscriber Server}
\newacronym{html5}{HTML5}{Hypertext Markup Language Version 5}
\newacronym{http}{HTTP}{Hypertext Transfer Protocol}

\newacronym{iad}{IAD}{Integrated Access Device}
\newacronym{iana}{IANA}{Internet Assigned Numbers Authority}
\newacronym{ibcf}{IBCF}{Interconnection Border Control Function}
\newacronym{ibgf}{IBGF}{Interconnection Border Gateway Function}
\newacronym{ice}{ICE}{Interactive Connectivity Establishment}
\newacronym{i-cscf}{I-CSCF}{Interrogating-CSCF}
\newacronym{iesg}{IESG}{Internet Engineering Steering Group}
\newacronym{ietf}{IETF}{Internet Engineering Task Force}
\newacronym{imei}{IMEI}{International Mobile Equipment Identity}
\newacronym{ims}{IMS}{IP Multimedia Subsystem}
\newacronym{imscm}{IMSCM}{IMS-Client-Manager}
\newacronym{imsi}{IMSI}{International Mobile Subscriber Identity}
\newacronym{imssf}{IM-SSF}{IP Multimedia Service Switching Function}
\newacronym{imt}{IMT}{International Mobile Telecomunication}
\newacronym{inap}{INAP}{Intelligent Network Application Part}
\newacronym{ip}{IP}{Internet Protocol}
\newacronym{ip-can}{IP-CAN}{Internet Protocol- Connectivity Access Network}
\newacronym{ipdv}{IPDV}{IP Delay Variation}
\newacronym{iper}{IPER}{IP Packet Error Ratio}
\newacronym{iplr}{IPLR}{IP Packet Loss Ratio}
\newacronym{iptd}{IPTD}{IP Transfer Delay}
\newacronym{iptv}{IPTV}{Internet Protocol Television}
\newacronym{isc}{ISC}{IMS Service Control}
%\newacronym{isdn}{ISDN}{Integrated Services Digital Network}
\newacronym{iso/osi}{ISO/OSI}{Open System Interconnection/International Organization for Standardization}
\newacronym{isup}{ISUP}{ISDN User Part}
\newacronym{itu}{ITU}{International Telecommunication Union}
\newacronym{itu-t}{ITU-T}{International Telecommunications Union-Telecommunication Standardization Sector}
\newacronym{it}{IT}{Informationstechnologie}

\newacronym{json}{JSON}{JavaScript Object Notation}

\newacronym{ki}{KI}{Künstliche Intelligenz}

\newacronym{la}{LA}{Location Area}
\newacronym{label}{API}{Application Programming Interface}
\newacronym{lcp}{LCP}{Link Control Protocol}
\newacronym{llc}{LLC}{Logical Link Control}
\newacronym{lp}{LP}{Linear Prediction}
\newacronym{lte}{LTE}{Long Term Evolution}

\newacronym{m2m}{M2M}{Machine to Machine Communication}
\newacronym{mac}{MAC}{Medium Access Control}
\newacronym{mac_encryption}{MAC}{Message Authentication Code (encryption context)}
\newacronym{map}{MAP}{Mobile Application Part}
\newacronym{mcf}{MCF}{Media Control Function}
\newacronym{mdct}{MDCT}{Modified Discrete Cosine Transform}
\newacronym{mdf}{MDF}{Media Delivery Function}
\newacronym{mfv}{MFV}{Mehrfach Frequenz Wahlverfahren/Tonwahl}
\newacronym{mgc}{MGC}{Media Gateway Controler}
\newacronym{mgcf}{MGCF}{Media Gateway Control Function}
\newacronym{mgw}{MGW}{Media Gateway Function}
\newacronym{mime}{MIME}{Multipurpose Internet Mail Extensions}
\newacronym{mm}{MM}{Mobility Management}
\newacronym{mme}{MME}{Mobile Management Entity}
\newacronym{mos}{MOS}{Mean Opinion Score}
\newacronym{mrf}{MRF}{Multimedia Resource Function}
\newacronym{mrfc}{MRFC}{Multimedia Resource Function Controller}
\newacronym{mrfp}{MRFP}{Multimedia Resource Function Processor}
\newacronym{ms}{MS}{Mobile Station}
\newacronym{msc}{MSC}{Mobile Switching Centre}
\newacronym{msisdn}{MSISDN}{Mobile Subscriber ISDN Number}

\newacronym{n-pvr}{N-PVR}{Network-Personal Video Recorder}
\newacronym{napt}{NAPT}{Network Address and Port Translation}
\newacronym{nas}{NAS}{Non-Access Stratum}
\newacronym{nas_server}{NAS}{Network Access Server}
\newacronym{nass}{NASS}{Network Attachment Subsystem}
\newacronym{nat}{NAT}{Network Address Translation}
\newacronym{nb}{NB}{Narrowband}
\newacronym{nfv}{NFV}{Network Function Virtualization}
\newacronym{ngn}{NGN}{Next Generation Network}
\newacronym{nni}{NNI}{Network-Network Interface}
\newacronym{nodeb}{NodeB}{Funkbasisstation im UTRAN}
\newacronym{nsapi}{NSAPI}{Network Service Access Point Identifier}
\newacronym{ntba}{NTBA}{Network Termination for ISDN Basic rate Access / Netzterminator Basisanschluss}

\newacronym{osa}{OSA}{Open Service Access}
\newacronym{osascs}{OSA-SCS}{Open Service Access - Service Capability Server}
\newacronym{osgi}{OSGi}{Open Services Gateway initiative}
\newacronym{osi}{OSI}{Open System Interconnection}
\newacronym{ott}{OTT}{Over The Top Anwendungen}

\newacronym{p-cscf}{P-CSCF}{Proxy Call Session Control Function}
\newacronym{pcc}{PCC}{Policy and Charging Control}
\newacronym{pcef}{PCEF}{Policy Control Enforcement Function}
\newacronym{pch}{PCH}{Paging Channel}
\newacronym{pcrf}{PCRF}{Policy and Charging Rules Function}
\newacronym{pcu}{PCU}{Packet Control Unit}
\newacronym{pdn}{PDN}{Public Data Network, Packet Data Network}
\newacronym{pdn-gw}{PDN-GW}{Packet Data Network Gateway}
\newacronym{pelr}{PELR}{Packet Error Loss Rate}
\newacronym{pes}{PES}{PSTN/ISDN Emulation Subsystem}
\newacronym{pesq}{PESQ}{Perceptual Evaluation of Speech Quality}
\newacronym{pgw}{PGW}{Packet Data Network Gateway}
\newacronym{php}{PHP}{PHP: Hypertext Preprocessor}
\newacronym{plmn}{PLMN}{Public Land Mobile Network}
\newacronym{plr}{PLR}{Packet Loss Rate}
\newacronym{pnai}{PNAI}{Personal Network Administration Interface}
\newacronym{polqa}{POLQA}{Perceptual Objective Listening Quality Assessment}
\newacronym{pots}{POTS}{Plain Old Telephone System}
\newacronym{ppv}{PPV}{Pay-Per-View}
\newacronym{ps}{PS}{Packet Switched, Location Probability}
\newacronym{pss}{PSS}{PSTN/ISDN Simulation Subsystem}
\newacronym{pstn}{PSTN}{Public Switched Telephone Network}

\newacronym{qci}{QCI}{QoS Class Identifier}
\newacronym{qoe}{QoE}{Quality of Experience}
\newacronym{qos}{QoS}{Quality of Service}

\newacronym{ra}{RA}{Routing Area, Random mode request information field}
\newacronym{rab}{RAB}{Radio Access Bearer, Access Burst}
\newacronym{rat}{RAT}{Radio Access Technology}
\newacronym{rach}{RACH}{Random Access Channel}
\newacronym{racs}{RACS}{Resource Admission Control Subsystem}
\newacronym{radius}{RADIUS}{Remote Authentication Dial In User Service}
\newacronym{rcs}{RCS}{Rich Communication Suite}
\newacronym{rlc}{RLC}{Radio Link Control}
\newacronym{rnc}{RNC}{Radio Network Controller}
\newacronym{rpe-ltp}{RPE-LTP}{Regular Pulse Excitation - Long Term Prediction}
\newacronym{rr}{RR}{Radio Resources}
\newacronym{rrc}{RRC}{Radio Resource Control}
\newacronym{rtc}{RTC}{Real Time Communication}
\newacronym{rtcp}{RTCP}{RTP Control Protocol}
\newacronym{rtp}{RTP}{Realtime Transport Protocol}
\newacronym{rtsp}{RTSP}{Real-Time Streaming Protocol}
\newacronym{rtt}{RTT}{Round Trip Time}

\newacronym{sacch}{SACCH}{Slow Associated Control Channel}
\newacronym{sai}{SAI}{Service Attachment Information}
\newacronym{sapi}{SAPI}{Service Access Point Identifier}
\newacronym{scf}{SCF}{Service Control Function}
\newacronym{sch}{SCH}{Synchronisation Channel}
\newacronym{s-cscf}{S-CSCF}{Serving-CSCF}
\newacronym{sctp}{SCTP}{Stream Control Transmission Protocol}
\newacronym{sd}{SD}{Standard Definition}
\newacronym{sdcch}{SDCCH}{Stand-Alone Dedicated Control Channel}
\newacronym{sdf}{SDF}{Service Discovery Function}
\newacronym{sdma}{SDMA}{Space Division Multiple Access}
\newacronym{sdp}{SDP}{Session Description Protocol}
\newacronym{servgw}{ServGW}{Serving Gateway}
\newacronym{sgsn}{SGSN}{Serving GPRS Support Node}
\newacronym{sgw}{S-GW}{Signaling Gateway}
\newacronym{sigtran}{SIGTRAN}{ZGSNr.7 Signaling Transport Protocol Suite}
\newacronym{silk}{SILK}{SILK Speech codec}
\newacronym{sip}{SIP}{Session Initiation Protocol}
\newacronym{sipas}{SIP AS}{SIP Application Server}
\newacronym{slf}{SLF}{Subscriber Location Function}
\newacronym{sms}{SMS}{Short Message Service}
\newacronym{srtp}{SRTP}{Secure Real-Time Transport Protocol}
\newacronym{ss}{SS}{Supplementary Service}
\newacronym{sscon}{SSCON}{Session Controller}
\newacronym{ssf}{SSF}{Service Selection Function}
\newacronym{ssi}{SSI}{Service Selection Information}
\newacronym{ssrc}{SSRC}{Synchronization Source}
\newacronym{stun}{STUN}{Session Traversal Utilities for NAT}
\newacronym{swb}{SWB}{Super-Wideband}
\newacronym{smd}{SMD}{State Machine Diagram}

\newacronym{tch}{TCH}{Traffic Channel}
\newacronym{tcp}{TCP}{Transmission Control Protocol}
\newacronym{td}{TD}{Transfer Delay}
\newacronym{tdd}{TDD}{Time Division Duplex}
\newacronym{tispan}{TISPAN}{Telecommunications and Internet converged Services and Protocols for Advanced Networking}
\newacronym{tmsi}{TMSI}{Temporary Mobile Subscriber Identity}
\newacronym{tn}{TN}{Termination Node, Timeslot Number}
\newacronym{toc}{TOC}{Table-Of-Contents}
\newacronym{turn}{TURN}{Traversal Using Relays around NAT}
\newacronym{tv}{TV}{Television}
\newacronym{tsr}{TSR}{Traffic Sign Recognition}


\newacronym{ua}{UA}{User Agent}
\newacronym{udp}{UDP}{User Datagram Protocol}
\newacronym{ue}{UE}{User Equipment}
\newacronym{umts}{UMTS}{Universal Mobile Telecommunications System}
\newacronym{uni}{UNI}{User-Network Interface}
\newacronym{up}{UP}{User Plane}
\newacronym{upsf}{UPSF}{User Profile Server Function}
\newacronym{uri}{URI}{Uniform Resource Identifier}
\newacronym{url}{URL}{Uniform Resource Locator}
\newacronym{utran}{UTRAN}{Universal Terrestrial Radio Access Network}
\newacronym{ucd}{UCD}{Use Case Diagram}

\newacronym{vad}{VAD}{Voice Activity Dedection}
\newacronym{vbr}{VBR}{Variable Bit  Rate}
\newacronym{vgw}{VGW}{Voice over IP-Gateway}
\newacronym{vlc}{VLC}{VideoLan Client}
\newacronym{vod}{VoD}{Video on Demand}
\newacronym{voip}{VoIP}{Voice Over IP}
\newacronym{volga}{VoLGA}{Voice over LTE via Generic Access}
\newacronym{volte}{VoLTE}{Voice over LTE}

\newacronym{w3c}{W3C}{World Wide Web Consortium}
\newacronym{waf}{WAF}{WebRTC Authentication Function}
\newacronym{wb}{WB}{Wideband}
\newacronym{webrtc}{WebRTC}{Web Real-Time Communication}
\newacronym{wlan}{WLAN}{Wireless Local Area Network}
\newacronym{wqsf}{WQSF}{WebRTC QoS Signalling Function}
\newacronym{wwsf}{WWSF}{WebRTC Web Server Function}

\newacronym{xdsl}{xDSL}{any Digital Subscriber Line}
\newacronym{xml}{XML}{Extended MarkupLanguage}

\newacronym{zgs nr.7}{ZGS Nr.7}{Zentrales Zeichengabesystem Nr.7}
\newacronym{zd}{ZD}{Zustandsdiagramm}

\newacronym{ann}{ANN}{Artificial Neural Network}
\newacronym{mse}{MSE}{Mean Squared Error}
\newacronym{ea}{EA}{Evolutionäre Algorithmen}
\makenoidxglossaries % index acronyms
% ----------------------------------------------------

\begin{document}
% ----------------------------------------------------------------------------------------------------------
% Titelseite
% ----------------------------------------------------------------------------------------------------------
\newgeometry{margin=0.5in}
\thispagestyle{empty}
\begin{titlepage}
	\begin{center}
		\begin{figure}[h]
			\raggedleft
			\graphicspath{ {Images/} }
			\includegraphics[scale=0.4]{logo_wbh.png}
		\end{figure}

		\vspace*{2cm}
		\Large \textbf{Studiengang:}\\
		\large \textbf{Embedded Systems and Digital Technologies}\\

		\vspace*{2cm}
		\LARGE \textbf{B-Aufgabe}\\
		\vspace*{0.5cm}
		\large B-TBU01-XX5-N01\\
		\vspace*{1cm}
		\Large \textbf{Technologiebasierte Unternehmensgründung}\\

		\vspace*{2cm}
		\vfill

		\normalsize
		\newcolumntype{x}[1]{>{\raggedleft\arraybackslash\hspace{0pt}}p{#1}}
		\begin{tabular}{x{6cm}p{7.5cm}}
			\rule{0mm}{5ex}\textbf{Student:} & Kilian Vogler \newline kilian.vogler@gmail.com \\
			\rule{0mm}{5ex}\textbf{Matrikelnummer:} & 861105 \\
			\rule{0mm}{5ex}\textbf{Abgabedatum:} & 03.10.2025 \\
		\end{tabular}
	\end{center}
\end{titlepage}
\pagebreak
\restoregeometry

% ----------------------------------------------------------------------------------------------------------
% Verzeichnisse
% ----------------------------------------------------------------------------------------------------------

\onehalfspacing
\pagenumbering{Roman}
\tableofcontents %print the table of contents
\cleardoublepage

% ----------------------------------------------------------------------------------------------------------
% Inhalt
% ----------------------------------------------------------------------------------------------------------
% Abstände Überschrift

% Kopfzeile
\renewcommand{\sectionmark}[1]{\markright{#1}}
\renewcommand{\subsectionmark}[1]{}
\renewcommand{\subsubsectionmark}[1]{}

\onehalfspacing
\renewcommand{\thesection}{\arabic{section}}
\renewcommand{\theHsection}{\arabic{section}}
\setcounter{section}{0}
\pagenumbering{arabic}

% ----------------------------------------------------------------------------------------------------------
% Kapitel: Aufgabenstellung B-EBF01-XX6
% ----------------------------------------------------------------------------------------------------------
\section*{Aufgabenstellung B-EBF01-XX6}
\addcontentsline{toc}{section}{\protect\numberline{}Aufgabenstellung B-EBF01-XX6}
text
\cleardoublepage
text
\cleardoublepage

% ----------------------------------------------------------------------------------------------------------
% Kapitel 1: Projektvorgehen, Organisation & rechtlicher Rahmen
% ----------------------------------------------------------------------------------------------------------
\section{Projektvorgehen, Organisation \& rechtlicher Rahmen}
\label{sec:projektvorgehen}

\subsection{(1a) Vorgehen, Beteiligte, Organisations-/Arbeitsstruktur, Standort}
\begin{aufgabenstellung}
Stellen Sie dar, wie Sie dieses Projekt angehen, insbesondere in Bezug auf Beteiligte, Organisations-/Arbeitsstruktur und Standort. Verwenden Sie hierzu Stärken-Schwächen-Analysen oder auch Nutzwertanalysen. \textbf{(8 Punkte)}
\end{aufgabenstellung}

\vspace*{5mm}

\subsubsection{Zielbild \& Annahmen}
% Kurz die Projektziele, Scope, Randbedingungen und Erfolgskriterien skizzieren.

\subsubsection{Stakeholderanalyse}
% Primäre/sekundäre Stakeholder, Rollen, Erwartungshaltungen, Einfluss/Macht
\begin{itemize}
  \item \textbf{Primär:} Kunde, Endnutzer, Projektauftraggeber, Kernteam
  \item \textbf{Sekundär:} Lieferanten, Partner, Regulierungsbehörden, interne Supportfunktionen
\end{itemize}

\subsubsection{Organisations- \& Arbeitsstruktur}
% Organigramm (Linie/Matrix), RACI, Entscheidungs- und Eskalationswege
% Hinweis auf iteratives/vs. stage-gate Vorgehen, Tools, Meetings, KPIs

\subsubsection{Standortwahl}
% Kriterien (Nähe Kunde, Talente, Kosten, Regulierung, IP-Schutz, Logistik, Zeitzonen, Förderungen)

\subsubsection{Stärken-Schwächen-Analyse (SWOT)}
% SWOT auf Unternehmensebene oder Projekt-Setup
\begin{table}[htb!]
\centering
\caption{SWOT-Analyse Projektsetup}
\begin{tabular}{p{0.45\linewidth} p{0.45\linewidth}}
\textbf{Stärken (S)} & \textbf{Schwächen (W)} \\
\hline
% ...
& \\
\textbf{Chancen (O)} & \textbf{Risiken (T)} \\
\hline
% ...
& \\
\end{tabular}
\end{table}

\subsubsection{Nutzwertanalyse (z.\,B. Standort- oder Organisationsoptionen)}
% Kriterien, Gewichtungen, Bewertungsskala, Gesamtnutzen je Option
\begin{table}[htb!]
\centering
\caption{Nutzwertanalyse Standort-/Organisationsoptionen}
\begin{tabular}{l r r r r}
\textbf{Kriterium} & \textbf{Gew.} & \textbf{Option A} & \textbf{Option B} & \textbf{Option C} \\
\hline
% Beispiel: Talentpool & 0.25 & 4 & 3 & 5
% ...
\multicolumn{2}{l}{\textbf{Gesamtnutzen}} & & & \\
\end{tabular}
\end{table}

\paragraph{Begründung}
% Detaillierte Begründung der getroffenen Wahl auf Basis der Analysen.

\clearpage

\subsection{(1b) Rechtlicher Rahmen (Organisationsform \& Patente)}
\begin{aufgabenstellung}
Wie sieht ein möglicher rechtlicher Rahmen für das Projekt aus (Organisationsform, Patentsituation)? Ergänzen Sie dabei die gemachten Angaben nach Ihren Bedürfnissen. \textbf{(8 Punkte)}
\end{aufgabenstellung}

\vspace*{5mm}

\subsubsection{Organisations-/Gesellschaftsform}
% Optionen (z.B. GmbH, AG, JV), Haftung, Kapitalbedarf, Governance, Steueraspekte

\subsubsection{IP-/Patentsituation}
% Freedom-to-Operate, Prior-Art-Suche, Schutzrechte (Patente, Marken, Designs), Lizenzstrategien, NDA, Arbeitnehmererfindungen

\subsubsection{Compliance \& Regulatorik}
% Branchenstandards, Exportkontrollen, Datenschutz, Produktsicherheit, Zertifizierungen

\paragraph{Begründung}
% Warum diese Rechts-/Organisationsform und IP-Strategie zum Vorhaben passt.


% ----------------------------------------------------------------------------------------------------------
% Kapitel 2: Marktanalyse & Eintrittsstrategie
% ----------------------------------------------------------------------------------------------------------
\section{Marktanalyse \& Eintrittsstrategie}
\label{sec:marktanalyse}

\subsection{(2a) Marktanalyse \& kritische Erfolgsfaktoren}
\begin{aufgabenstellung}
Entwickeln Sie auf Basis einer umfassenden Marktanalyse (Porter 5 Forces, Umgebungsanalyse, Value Chain Analysis o.\,Ä.) kritische Erfolgsfaktoren für das Projekt. \textbf{(16 Punkte)}
\end{aufgabenstellung}

\vspace*{5mm}

\subsubsection{Porter Five Forces}
% Käufermacht, Lieferantenmacht, Ersatzprodukte, Markteintrittsbarrieren, Rivalität
% ggf. Abbildung
% \begin{figure}[htb!]\centering\includegraphics[width=0.9\linewidth]{Images/porter.pdf}\caption{Porter Five Forces}\label{fig:porter}\end{figure}

\subsubsection{Umgebungsanalyse (z.\,B. PESTEL)}
% Political, Economic, Social, Technological, Environmental, Legal

\subsubsection{Value Chain Analysis}
% Primär- und Unterstützungsaktivitäten, Wo entsteht differenzierender Wert?

\subsubsection{Kritische Erfolgsfaktoren (KEF)}
\begin{itemize}
  \item KEF~1: % ...
  \item KEF~2: % ...
  \item KEF~3: % ...
\end{itemize}

\paragraph{Begründung}
% Herleitung der KEF direkt aus den Analysen, Messgrößen/KPIs definieren.

\subsection{(2b) Eintrittsoptionen, Bewertung, Empfehlung \& Back-up}
\begin{aufgabenstellung}
Stellen Sie mögliche Eintrittsoptionen in den Markt dar und bewerten Sie diese entsprechend. Was ist Ihre Empfehlung und warum? Was planen Sie als Back-up-Lösung? \textbf{(9 Punkte)}
\end{aufgabenstellung}

\vspace*{5mm}

\subsubsection{Eintrittsoptionen}
% Greenfield, Partnerschaft/Allianz, Lizenz/Franchising, Joint Venture, M\&A, Plattform/Marketplace

\subsubsection{Bewertung (Nutzwert-/Risiko-Nutzen-Matrix)}
\begin{table}[htb!]
\centering
\caption{Bewertung Markteintrittsoptionen}
\begin{tabular}{l r r r r}
\textbf{Kriterium} & \textbf{Gew.} & \textbf{Option 1} & \textbf{Option 2} & \textbf{Option 3} \\
\hline
% Time-to-Market, CAPEX/OPEX, regulatorisches Risiko, Skalierbarkeit, Kontrolle, Marge
\multicolumn{2}{l}{\textbf{Gesamtscore}} & & & \\
\end{tabular}
\end{table}

\subsubsection{Empfehlung}
% Gewählte Option, Roadmap, Meilensteine, Erfolgskriterien

\subsubsection{Back-up-Strategie}
% Alternativpfad bei Scheitern, Abbruch-/Pivot-Kriterien, Verwässerungs- und Kostenrahmen

\clearpage

% ----------------------------------------------------------------------------------------------------------
% Kapitel 3: Marketing-/Entwicklungsplan \& Positionierung
% ----------------------------------------------------------------------------------------------------------
\section{Marketing-/Entwicklungsplan \& Positionierung}
\label{sec:marketing_plan}

\subsection{(3a) Plan auf Basis Positionierung \& Segmentierung; Portfolioszenarien}
\begin{aufgabenstellung}
Erstellen Sie auf Grundlage der in Aufgabe 2 ermittelten Marktpositionierung und auf Basis einer Zielgruppensegmentierung einen Marketing-/Entwicklungsplan. Zeigen Sie mögliche Portfolioszenarien (Produkt, Preis, Service) auf. \textbf{(15 Punkte)}
\end{aufgabenstellung}

\vspace*{5mm}

\subsubsection{Zielgruppensegmentierung}
% Segmente, Größe/Wachstum, Needs/Jobs-to-be-done, Zahlungsbereitschaft, Priorität
\begin{table}[htb!]
\centering
\caption{Segmentübersicht}
\begin{tabular}{l r r l}
\textbf{Segment} & \textbf{Volumen} & \textbf{Wachstum} & \textbf{Kernbedürfnisse} \\
\hline
% ...
\end{tabular}
\end{table}

\subsubsection{Positionierung}
% Zielposition, Wettbewerbsdifferenzierung, Reason-to-Believe

\subsubsection{Marketing-/Go-to-Market-Plan}
% Kanäle, Taktiken, Funnel, Content, Partnerschaften, Budget/KPIs, Zeitplan

\subsubsection{Entwicklungsplan (Roadmap)}
% Releases, MVP->V1->V2, Lernziele/Experimente, Qualitäts-/Zulassungsschritte
% \begin{figure}[htb!]\centering\includegraphics[width=\linewidth]{Images/roadmap.pdf}\caption{Produkt-/Tech-Roadmap}\label{fig:roadmap}\end{figure}

\subsubsection{Portfolioszenarien (Produkt, Preis, Service)}
\begin{table}[htb!]
\centering
\caption{Portfolioszenarien}
\begin{tabular}{l l l l}
\textbf{Szenario} & \textbf{Produktbündel} & \textbf{Preislogik} & \textbf{Servicelevel} \\
\hline
% Basis & ... & Subscription/One-off & Self-Service
% Premium & ... & Value-based & Managed
% Enterprise & ... & Custom & SLA/Support 24/7
\end{tabular}
\end{table}

\paragraph{Begründung}
% Warum diese Segmente/Positionierung/Portfolio; Bezug zu KEF und Marktanalyse.

\subsection{(3b) Kundenerreichung \& USP-Statement}
\begin{aufgabenstellung}
Wie stellen Sie sicher, dass Sie Ihre Kunden erreichen? Formulieren Sie eine kurze, zusammenfassende Darstellung des Alleinstellungsmerkmals zur gewünschten Positionierung auf Basis der analysierten Segmentierung. \textbf{(5 Punkte)}
\end{aufgabenstellung}

\vspace*{5mm}

\subsubsection{Kanalstrategie \& Abdeckung}
% Kanal-Mix, Reichweite vs. Effizienz, Tests/Iterationen, KPI-Deckung

\subsubsection{USP (Kurzstatement)}
% Vorlage:
% Für [Zielsegment], die/der [Bedarf/Problem], bietet [Produkt/Angebot] [Kerndifferenzierung],
% im Gegensatz zu [Hauptalternative], weil [Beweis/Reason-to-Believe].

\paragraph{Begründung}
% Warum diese Kanäle/Message-Market-Fit funktionieren werden.


% ----------------------------------------------------------------------------------------------------------
% Kapitel 4: Finanzplanung \& Finanzierung
% ----------------------------------------------------------------------------------------------------------
\section{Finanzplanung \& Finanzierung}
\label{sec:finanzplanung}

\begin{aufgabenstellung}
Stellen Sie anhand Ihrer Finanzplanung (z.\,B. GuV, Cashflow, Bilanz in Tabellenform) dar, welche Mindestinvestitionssumme von Dritten eingebracht werden muss. Erklären Sie in kurzen Worten, welche Finanzierungsmöglichkeiten Ihnen zur Verfügung stehen und welche Vor- und Nachteile diese haben. \textbf{(25 Punkte)}
\end{aufgabenstellung}

\vspace*{5mm}

\subsection{Plan-GuV (3--5 Jahre)}
\begin{table}[htb!]
\centering
\caption{Plan-GuV}
\begin{tabular}{l r r r r}
\textbf{Position} & \textbf{Jahr 1} & \textbf{Jahr 2} & \textbf{Jahr 3} & \textbf{Jahr 4} \\
\hline
Umsatz & & & & \\
COGS & & & & \\
\textbf{Bruttoergebnis} & & & & \\
OPEX (F\&E, Vertrieb, Allg.) & & & & \\
EBITDA & & & & \\
Abschreibungen & & & & \\
EBIT & & & & \\
Zinsen/Steuern & & & & \\
\textbf{Jahresüberschuss} & & & & \\
\end{tabular}
\end{table}

\subsection{Cashflow-Plan}
\begin{table}[htb!]
\centering
\caption{Cashflow}
\begin{tabular}{l r r r r}
\textbf{Position} & \textbf{Jahr 1} & \textbf{Jahr 2} & \textbf{Jahr 3} & \textbf{Jahr 4} \\
\hline
Operativer Cashflow & & & & \\
Investitions-Cashflow & & & & \\
Finanzierungs-Cashflow & & & & \\
\textbf{Netto-Cashflow} & & & & \\
Kassenbestand (Ende) & & & & \\
\end{tabular}
\end{table}

\subsection{Bilanz (Stichtag)}
\begin{table}[htb!]
\centering
\caption{Bilanz (verkürzt)}
\begin{tabular}{l r l r}
\multicolumn{2}{c}{\textbf{Aktiva}} & \multicolumn{2}{c}{\textbf{Passiva}} \\
\hline
Anlagevermögen & \hspace{6em} & EK & \hspace{6em} \\
Umlaufvermögen & & FK kurz/lang & \\
Kassenbestand & & Rückstellungen & \\
\textbf{Summe Aktiva} & & \textbf{Summe Passiva} & \\
\end{tabular}
\end{table}

\subsection{Kapitalbedarf \& Mindestinvestitionssumme}
% Herleitung: kumulierter negativer Cashflow + Sicherheitszuschlag + Working Capital
% Darstellung der Liquiditätslücke und Zeitpunkt(e) der Finanzierungstranchen.

\subsection{Finanzierungsoptionen: Vor- \& Nachteile}
% \begin{table}[htb!]
% \centering
% \caption{Finanzierungsinstrumente im Vergleich}
% \begin{tabular}{l l l}
% \textbf{Instrument} & \textbf{Vorteile} & \textbf{Nachteile} \\
% \hline
% Eigenkapital (VC/BA) & % Smart Money, Risiko-Sharing
% & % Verwässerung, Governance-Auflagen \\
% Venture Debt-Bankdarlehen & & \\
% Fördermittel-Zuschüsse & & \\
% Mezzanine-Convertible & & \\
% Crowdfunding & & \\
% \end{tabular}
% \end{table}

\paragraph{Begründung}
% Warum diese Struktur/Timing passend ist; Sensitivitäten/Downside-Case.

\clearpage

% ----------------------------------------------------------------------------------------------------------
% Kapitel 5: Ablehnung Hongkong – Ursachen, Risiken, Präventionsmaßnahmen
% ----------------------------------------------------------------------------------------------------------
\section{Ablehnung Hongkong – Ursachen, Risiken, Präventionsmaßnahmen}
\label{sec:hongkong_risiken}

\begin{aufgabenstellung}
Sie haben den Zuschlag für das Projekt in Hongkong leider nicht bekommen (nur 2. Sieger) und stehen vor einem potenziellen finanziellen Scherbenhaufen – erklären Sie mögliche Gründe der Ablehnung (mindestens 3). Welche zusätzlichen Risiken können während des Bieterverfahrens aufgetreten sein? Erklären Sie, welche möglichen Risikominimierungsmaßnahmen Sie daher für Ihr Projekt besser vorab etabliert hätten?
\end{aufgabenstellung}

\vspace*{5mm}

\subsection{Mögliche Ablehnungsgründe (mind. 3)}
\begin{itemize}
  \item Grund~1: % ...
  \item Grund~2: % ...
  \item Grund~3: % ...
  \item (optional) weitere Gründe
\end{itemize}

\subsection{Zusätzliche Risiken im Bieterverfahren}
% z.B. Währungs-/Länderrisiko, Partner-/Lieferantenabhängigkeit, Compliance, Scope Creep, IP-/Datenübernahme, Terminrisiken

\subsection{Risikominimierung (präventiv/proaktiv)}
% \begin{table}[htb!]
% \centering
% \caption{Risikoregister (Auszug)}
% \begin{tabular}{l l l l}
% \textbf{Risiko} & \textbf{Einfluss} & \textbf{Wahrsch.} & \textbf{Maßnahmen (präventiv/reagierend)} \\
% \hline
% % ...
% \end{tabular}
% \end{table}

\paragraph{Begründung}
% Welche Maßnahmen hätten die Ablehnungs-/Schadenwahrscheinlichkeit reduziert; Lessons Learned \& nächste Schritte.

\clearpage

\cleardoublepage

\appendix
\pagenumbering{Roman}
\setcounter{page}{1}

% Redefine \chapter* to not cause a page break
\makeatletter
\renewcommand\chapter{\@startsection{chapter}{0}{\z@}%
	{-3.5ex \@plus -1ex \@minus -.2ex}%
	{2.3ex \@plus.2ex}%
	{\normalfont\Large\bfseries}}
\makeatother

\begingroup
\raggedright
\printbibliography
\endgroup

\nopagebreak
\printnoidxglossary[type=\acronymtype] % print the list of abbreviations

\nopagebreak
\listoffigures % print the list of figures

% \nopagebreak
% \listoftables % print the list of tables

\nopagebreak
\lstlistoflistings % print the list of source codes

% \clearpage
% \makedeclarationofauthorship % print the declaration of authorship

\end{document}